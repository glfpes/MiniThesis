\documentclass{article}
\usepackage{scalerel}
\usepackage{stackengine}
\usepackage{verbatimbox}
\usepackage{boxhandler}
\parindent 0in\parskip 1em
\newlength\dotscale

\setlength{\dotscale}{.4ex}         % DOTSIZE<---CAN CHANGE THESE
\def\newdot{\scalerel*{.}{\rule[-\dotscale]{1ex}{\dotscale}}}
\def\newdot{\scalerel*{.}{\rule{1ex}{\dotscale}}}

% FOR CLOSE-BOXING ARGUMENTS
\fboxsep=0pt\fboxrule=.6pt

\def\rl{\rule[-.3pt]{2ex}{.6pt}}
\def\ste{\textsf{stackengine}}
\def\bs{\texttt{\char'134}}
\let\vb\verb
\reversemarginpar
\marginparwidth 1.6in
\newcommand\margtt[1]{\marginpar{\hfill\ttfamily#1}}
\newcommand\margcmd[1]{\marginpar{\hfill\ttfamily\char'134#1}}
\newcommand\cmd[1]{\texttt{\char'134#1}}
\def\lapmark{\rule[-.3\baselineskip]{.1ex}{\baselineskip}}
\def\blmark{\rl\stackengine{-.9ex}{B}{\rule{.35ex}{0pt}L}{U}{l}{F}{T}{S}\rl}

\begin{document}

\begin{center}
\LARGE The {\ste} Package\\
\rule{0em}{.7em}\small Highly customized stacking of objects, insets,
baseline changes, \textit{etc.}\\
\rule{0em}{2.7em}\large Steven B. Segletes\\
steven.b.segletes.civ@mail.mil\\
\rule{0em}{1.7em}\today\\
\stackengineversionnumber%
\end{center}

%\tableofcontents

\sffamily%
A sectional index for this package is found in \S\ref{s:index}.

Note: See \S\ref{s:bc} for details on backward compatibility.
\rmfamily%

\section{Definitions and Terms}

Before I describe the various commands of this package, it is best for
you to familiarize yourself with a few terms as used in this package.

\subsection{Short Stacks and Long Stacks\label{s:st}}
\strutshortanchors{T}
\def\stacktype{L}
\Huge\setstackgap{S}{.4ex}
\setstackgap{L}{2.2ex}\normalsize 

In this package, ``Short'' refers to squeezing the extra space out of
glyphs comprising the stack, so that the inter-item gap is solely what
is set.  Likewise, ``Long'' refers to a stack kept at constant
inter-item baseline spacing.  Let's look at the following two
examples to see this clearly.

Notice the consistent inter-item gap on a short stack: 
\Huge
\Shortstack{{\fbox{t}} {\fbox{y}} {\fbox{k}} {\fbox{e}}}
\normalsize

Notice the consistent inter-baseline spacing on a long stack: 
\Huge
\Longstack{{{\rl t}} {{\rl y}} {{\rl k}} {{\rl e}}}
\normalsize\\

In this package, the default spacing (\textit{i.e.}, stacklength) for
the long stack, saved in \vb|\Lstackgap| is the value
\vb|\baselineskip|.  For short stacks, the default stacklength, stored in
\vb|\Sstackgap|, is 3pt, which corresponds to the same spacing found in
\TeX's \vb|\shortstack| command.  The default stack type constructed by
this package is short, denoted by the definition of \vb|\stacktype| as
``S'' (as opposed to ``L'' for long stacks).  These defaults can be
changed.

\subsection{Baseline}
\setstackgap{L}{.8\baselineskip}
\setstackgap{S}{4pt}

The baseline is the horizontal line on which the current (unstacked)
text is set.  For non-descending letters, the bottom of the letters sit
upon the baseline.  In this document, we will graphically use the symbol
{\blmark} to denote the current baseline of the text, as in

\blmark\rl\Shortunderstack{A B C}\rl\rl text continues here

\subsection{(Over)Stacks and Understacks\label{s:osus}}

\setstackgap{L}{.8\baselineskip}
\setstackgap{S}{4pt}
In general terms, a stack is built up from the baseline, while an
understack is built down from the baseline.  

So, for example,
\Longstack{a b c}
is a stack, while
\Longunderstack{a b c}
is an understack.

With this package, both stacks and understacks can be long or short,
depending on how the invocation is made.  Sometimes, in this
documentation, a stack will be called an ``overstack'' in order to avoid
confusion with ``understack.''

\subsection{Anchors\label{s:anch}}

The term ``anchor'' is used to denote an item in the stack whose
baseline does not change in the course of the stacking operation.  In
essence, stacking is done relative to the anchor.

For example, in \Shortstack{a b c}, the letter ``c'' is the anchor.  

In the understack \Shortunderstack{a b c}, ``a'' is the anchor.

\subsection{Alignment\label{s:sal}}

Anyone familiar with \LaTeX{} knows something about the horizontal
alignment settings ``l'', ``c'' and ``r''.  As pertains to a stack, we
refer to the alignment of the various rows of the stack, as shown in the
following examples.

\setstackgap{S}{3pt}\setstackgap{L}{12pt}%

{\centering
Left: \Shortstack[l]{abcd abc ab a}~~
Center: \Shortstack[c]{abcd abc ab a}~~
Right: \Shortstack[r]{abcd abc ab a}

}
The default alignment within a stack may be set with the
\vb|\stackalignment| definition, to possible values of \vb|l|,
\vb|c|, or \vb|r|, denoting left, center, and right alignments,
respectively.  In addition, a number of the stacking macros provide the
alignment as an argument or optional argument that can be specified at
time of invocation.

The {\ste} package defaults to center alignment, unless reset by way of
the \vb|\stackalignment| definition.


\section{Parameter Defaults, Modes, and Macros Provided by \ste}


\subsection{Parameter Defaults and How to Set Them\label{s:parms}}

There are several macros that serve to store package default
definitions.  When a stacking macro does NOT explicitly ask for one of the
quantities defined by these defaults, it will use the values of these
parameters to serve as the default value.  However, some stacking macros
may ask the user to specify an explicit value for any number of these
parameters in the calling arguments.  In that case, an explicit value
may be provided or else the macro holding the default value may
itself be provided as the argument.

Let us first consider the stackgap defaults, associated with short and
long stacks.  Those stackgaps are actually stored in the \vb|\def|'s

\vb|\Sstackgap|\\
\vb|\Lstackgap|

The 
\margcmd{Sstackgap}%
parameter \vb|\Sstackgap| is the default inter-item gap for any
stacking command that builds a short stack.  It defaults to 3pt.
Correspondingly,
\margcmd{Lstackgap}
\vb|\Lstackgap| is the inter-item baselineskip for
any stacking command that builds a long stack.  It defaults to
\vb|\baselineskip|.  To change these defaults, the following macro
is provided:
\margcmd{setstackgap}

\itshape
\vb|\setstackgap{S}{|inter-item stackgap\vb|}|\\
\vb|\setstackgap{L}{|inter-item baselineskip\vb|}|
\upshape

In \margcmd{stackgap}addition, there is the \vb|\stackgap| macro, which
may be invoked by the user.  This macro will look at the current value
of \vb|\stacktype| and will output either \vb|\Sstackgap| or
\vb|\Lstackgap|, depending on whether \vb|\stacktype| is defined as
``S'' (short) or ``L'' (long).

In addition to these lengths, there are several parameter definitions
that define default behaviors for those stacking macros when explicit
values of these settings are not requested.  The parameters, each set
with a \vb|\def| (or a \vb|\renewcommand|), are

\itshape
\vb|\def\stackalignment{l| or \vb|c| or \vb|r}|\\
\vb|\def\quietstack{T| or \vb|F}|\\
\vb|\def\useanchorwidth{T| or \vb|F}|\\
\vb|\def\stacktype{S| or \vb|L}|
\upshape

The parameter \vb|\stackalignment| \margcmd{stackalignment}defaults to
``c'' (center) and defines the default stacking alignment.  Other
options include ``l'' (left) and ``r'' (right).  See \S\ref{s:sal} for
more discussion of this parameter.

The parameter \vb|\quietstack| \margcmd{quietstack}defaults to ``F''
(false) and determines whether the result of a stacking operation is
suppressed from the output.  If the output is suppressed by setting this
parameter to ``T'', the most recent stacking operation may be recovered
with a \vb|\usebox{\stackedbox}|.  See \S\ref{s:qs} for more discussion
of this parameter.

The parameter \vb|\useanchorwidth| \margcmd{useanchorwidth}defaults to
``F'' (false) and indicates that, upon creation, the stack width should
be taken as the widest element of the stack.  When set to ``T'', the
stack width, upon creation, is instead taken as the width of the anchor.
Note that this parameter applies at the time of stack creation and
cannot be used to retroactively change the characteristics of an
existing stack.  See \S\ref{s:uaw} for more discussion of this
parameter.

The parameter \vb|\stacktype| \margcmd{stacktype}defaults to ``S''
(short stack) and defines the default type of stack to build.  When set
to ``L'', newly created stacks are, by default, long stacks.  Refer to
\S\ref{s:st} for a definition of these terms.

For macros that require explicit specification of some of these
parameters, the parameter token itself (\textit{e.g.},
\cmd{stackalignment}, \cmd{quietstack}, \textit{etc.}) may be provided
as the argument, if the default behavior is desired.

\subsection{Stacking Modes\label{s:sm}}

In addition to parameter definitions, there are several {\ste} mode
settings:

\itshape
\vb|\stackMath|\\
\vb|\stackText|\\
\vb|\strutlongstacks{T| or \vb|F}|\\
\vb|\strutshortanchors{T| or \vb|F}|\\
\vb|\setstackEOL{|end-of-line character\vb|}|
\upshape

By default, this package will assume that the arguments of the stacking
commands are text.  However, an invocation of \vb|\stackMath| 
\margcmd{stackMath}will
change the default so that {\ste} arguments will be processed in inline
math mode (so-called \vb|\textstyle| math).  Likewise, an invocation
of \vb|\stackText| 
\margcmd{stackText}will reset the default processing of arguments to
text mode.   Both of these modes are global in their scope, so that invocation, 
even within a macro, will continue to persist until later changed.  If
you would like to have a single application stacked in math mode, without
changing the default mode setting, one could use the \vb|\ensurestackMath{}| 
macro (see \ref{s:esm}).

By default, stacks made by {\ste} are set in unframed boxes that are 
flush with the top and bottom of the stack.  The \vb|\addstackgap| 
macro (\S\ref{s:asg}) is one way to add a buffer space above and below a stack.
Such an approach makes more sense for short stacks, where a constant
buffer size determines the gap between objects.  But for long stacks, a 
method involving \vb|\strut|'s makes more sense.  The mode 
\vb|\strutlongstacks|%
\margcmd{strutlongstacks}, when set true (\vb|T|), will automatically 
cause {\ste} to condition each row of a long stack with a \vb|\strut| 
during its construction (see \S\ref{s:ag} for more details).

The \vb|\strutshortanchors| 
\margcmd{strutshortanchors}macro (\S\ref{s:sa}) is used to set the mode of how
two items may be vertically combined into a split short-stack anchor.  
By default true (\vb|T|), a split short-stack anchor is set so that the 
middle of the stack gap falls halfway up the height of the \vb|\strut|.  
This will give the short-stack anchor the appearance of being split 
relative to text that sits atop the baseline.  When set false, however, 
any short-stack anchor that is constructed has its gap split across the 
baseline itself.

Until V3.1 of \ste, the \vb|\Longstack|, \vb|\Shortstack|, 
\vb|\Longunderstack|, and \vb|\Shortunderstack| macros used,
exclusively, space tokens as the end-of-line separators between
rows of these stacks.  Not only is such a syntax atypical of \LaTeX{}, 
but it can be inconvenient when the items within a stack row
also contain spaces.  The \vb|\setstackEOL| 
\margcmd{setstackEOL}macro allows the end-of-line (EOL)
character to be defined to a more convenient setting, such as \vb|\\|,
though any character will work.  There is a package option 
\vb|[usestackEOL]| \margtt{[usestackEOL]\rmfamily\\\raggedleft package
option} which sets the end-of-line symbol to be used when parsing stack
arguments to \vb|\\|.  When the stacking EOL is something other than a
space token, leading and trailing spaces are ignored in the row data for
the four cited stacking commands, as well as their derivatives 
\vb|\Centerstack| and \vb|\Vectorstack|.  (Note: other stacking commands
in the \vb|\stackon| / \vb|\stackunder| family will not ignore, but
retain spaces in their arguments, since there is no issue associated
with EOL seperators.)

Note that these five mode-setting commands are not themselves parameters
to be part of a {\ste} argument, but are instead to be issued as separate
commands prior to the invocation of a desired stacking macro.

\subsection{Stacking Macros}

\subsubsection{\cmd{stackengine}\label{s:stackeng}}

The basic (and most general) stacking macro provided by this package is
\vb|\stackengine|.  Nearly all other macros of this package merely
provide an abbreviated invocation form of this macro, tailored for a
particular application.  The syntax is:

\itshape
\vb|\stackengine{\Sstackgap| or
  \vb|\Lstackgap| or
  \vb|\stackgap| or stacklength%
  \vb|}|\\
\vb|            {|anchor\vb|}|\\
\vb|            {|item\vb|}|\\
\vb|            {O| or \vb|U}|\\
\vb|            {\stackalignment| or
  \vb|l| or
  \vb|c| or
  \vb|r}|\\
\vb|            {\quietstack| or \vb|T| or \vb|F}|\\
\vb|            {\useanchorwidth| or \vb|T| or \vb|F}|\\
\vb|            {\stacktype| or \vb|S| or \vb|L}|
\upshape

There are no optional arguments with \vb|\stackengine|.  The first
item is a stacking length, which can mean different things depending on
whether a short- or long-stack is being created, as discussed in
\S\ref{s:st}.  The second argument is the stack anchor, whose
baseline does not change.  The third argument is the item stacked
relative to the anchor.  The fourth argument is either an ``O'' (for a
normal or overstack) or a ``U'' (for an understack).  The fifth argument
denotes a left, center, or right alignment of the stack items with the
use of ``l'', ``c'', or ``r''.  The sixth argument is either a ``T'' or
``F'' to denote whether the resulting stack is NOT printed (``T''
denotes ``do NOT print'').  The seventh argument, also a ``T'' or an
``F'', denotes when ``T'' that the width of the whole stack is to be
taken as the width of the anchor.  Otherwise, the width of the stack
conforms to the width of the widest item in the stack.  The final
argument is defined as either an ``S'' or an ``L'' to denote whether a
short stack or a long stack is being requested.

While none of the arguments to \vb|\stackengine| have default values,
there are various definitions in the package that store default values.
When one wishes a stack to possess certain default properties, these
default definitions may be passed as the values for the respective
arguments to \vb|\stackengine|.  Arguments 1, 5, 6, 7, and 8 each
possess a corresponding parameter containing the default variable.  So,
for example,

\begin{verbatim}
\stackengine{\stackgap}{A}{BC}{O}{\stackalignment}
            {\quietstack}{\useanchorwidth}{\stacktype}
\end{verbatim}

will stack the letters ``BC'' over ``A'' using the default values of
\vb|\stackalignment|, \vb|\quietstack|, \vb|\useanchorwidth|,
and \vb|\stacktype|.  In addition, the stackgap will be set to
\vb|\Sstackgap| if \vb|\stacktype| is ``S'' or to \vb|\Lstackgap|
if \vb|\stacktype| is ``L''.  Alternately, the  same stack could be 
forced to be a long stack with right-alignment and be typeset using

\begin{verbatim}
\stackengine{\Lstackgap}{A}{BC}{O}{r}{F}{\useanchorwidth}{L}
\end{verbatim}
\vspace{-0.8ex}

While the stacking macros in the following sections offer the convenience 
of a condensed syntax, which can be useful within your document, it may be
preferable to use the \vb|\stackengine| macro to create your own
stacking macros, since one can achieve the stacking result without 
having to modify the package defaults.

\subsubsection{\cmd{stackon} and \cmd{stackunder}\label{s:sou}}

The \vb|\stackon| and \vb|\stackunder| macros are abbreviated forms
of \vb|\stackengine| in which the default values of parameters are
taken.  The only exception is the stacklength, which may be provided as
an optional argument.

\itshape
\vb|\stackon[|stacklength%
  \vb|]{|anchor%
  \vb|}{|item
  \vb|}|\\
\vb|\stackunder[|stacklength%
  \vb|]{|anchor%
  \vb|}{|item
  \vb|}|
\upshape

In the case of \vb|\stackon|, an (over)stack is created, whereas an
understack is created with \vb|\stackunder|.  With both of these
commands, the anchor is the first mandatory argument, and the item to be
stacked above or below it is the second mandatory argument.\vspace{-0.8ex}

\subsubsection{\cmd{Shortstack} and \cmd{Longstack}\label{s:sls}}

The \vb|\Shortstack| and \vb|\Longstack| commands are stacking commands
that allow more than two stacking rows to be specified at once.  The
items of a stack are provided, by default, in a space-separated list, as
follows:

\itshape
\vb|\Shortstack[|alignment%
  \vb|]{|item %
  \vb|{|it em%
  \vb|} {\|item%
  \vb|} $|item%
  \vb|$| ... anchor%
  \vb|}|\\
\vb|\Longstack[|alignment%
  \vb|]{|item %
  \vb|{|it em%
  \vb|} {\|item%
  \vb|} $|item%
  \vb|$| ... anchor%
  \vb|}|
\upshape

The first item is at the top of the stack, farthest from the baseline,
while the last item is the anchor of the stack.  If an item (the
contents of a single row) contains spaces, enclose the item in braces.
If the item ends with a macro, enclose the item in braces. 
If used in a mixed text/math mode (see \S\ref{s:mm} for more
details), math items may be enclosed within dollar-sign delimiters.  As of 
V3.2, these commands have been made robust and their arguments need
not be \verb|\protect|ed.

The alignment may be specified as an optional argument, with other
parameters taken by their defaults.  The stacklength is taken as
\vb|\stackgap|, which will depend on the value of \vb|\stacktype|.

To give an example of the braced syntax described above,
the stack\\
\vb|\Shortstack[r]{a {b c} {\P} $\beta^2$ ANCHOR}|
will produce
\Shortstack[r]{a {b c} {\P} $\beta^2$ ANCHOR}.  Note that the use of the
math delimiter (\vb|$|) in the \vb|\Shortstack| argument was needed
only if the stack contains both inline math as well as text elements.
If the stack is to contain only math-mode elements, then math arguments
can be made the default:\\
\vb|\stackMath|%
\vb|\Shortunderstack[c]{{\alpha} {\beta} {\gamma}}|\\
yielding
\stackMath
\Shortunderstack[c]{{\alpha} {\beta} {\gamma}}\.
\stackText

As of V3.1 of \ste, the user may define the the argument to 
\vb|\Longstack|, \vb|\Shortstack|, \vb|\Longunderstack|, and
\vb|\Shortunderstack| with a row separator different than a space.
The \vb|[usestackEOL]| package option will make the EOL character within
these stacking commands a \vb|\\| character instead of a space.  Here
are an example of equivalent stack arguments for two different EOL
characters:

Space EOL:~~~~\vb|{a {b c} {\P} $\beta^2$ ANCHOR}|\\
\vb|\\| EOL:~~~~~~~~\,\vb|{a \\ b c \\ \P \\ $\beta^2$ \\ ANCHOR}|

Note that leading and trailing spaces surrounding the \verb|\\| are ignored 
for rows in the second
case.  The \vb|\setstackEOL| macro allows any arbitrary character to be
set as the ``EOL'' character (for the purposes of parsing these
stacking macros).

\subsubsection{\cmd{Shortunderstack} and \cmd{Longunderstack}\label{s:slus}}

The \vb|\Shortunderstack| and \vb|\Longunderstack| commands are the
understacking equivalents of \vb|\Shortstack| and \vb|\Longstack|.

\itshape
\vb|\Shortunderstack[|alignment%
  \vb|]{|anchor item %
  \vb|{|it em%
  \vb|} {\|item%
  \vb|} $|item%
  \vb|$| ...%
  \vb|}|\\
\vb|\Longunderstack[|alignment%
  \vb|]{|anchor item %
  \vb|{|it em%
  \vb|} {\|item%
  \vb|} $|item%
  \vb|$| ...%
  \vb|}|
\upshape

The caveats to the specification of the space-separated list of items is
the same as that mentioned above for \vb|\Shortstack| and
\vb|\Longstack|.

Note that the items are still specified from the top downward.
In the case of understacks; however, that means that the anchor is the
first item specified, while the last item in the list is farthest below
the baseline.

\subsubsection{Top and bottom lapping macros\label{s:lapdef}}

\tclap{\smash{Top}}and \bclap{bottom}~lapping is achievable through the creation of various
long stacks.  The syntax is the same for all six of the macros provided.
The six names differentiate whether the lap is to the top (t) or the
bottom (b) of the calling point, and whether it should appear to the
left (l), centered (c), or to the right (r) of the calling location.
The long-stacklength may be provided as an optional argument.

\itshape
\vb|\tllap[|stacklength\vb|]{|item\vb|}|\\
\vb|\tclap[|stacklength\vb|]{|item\vb|}|\\
\vb|\trlap[|stacklength\vb|]{|item\vb|}|\\
\vb|\bllap[|stacklength\vb|]{|item\vb|}|\\
\vb|\bclap[|stacklength\vb|]{|item\vb|}|\\
\vb|\brlap[|stacklength\vb|]{|item\vb|}|
\upshape

See \S\ref{s:lap} for more information and examples of use for
these lapping commands.  If trying to remember the names of all these
lapping commands is too tedious for you, two functionally identical
macros have been introduced:

\itshape
\vb|\toplap[|stacklength\vb|]{|lap H-direction\vb|}{|item\vb|}|\\
\vb|\bottomlap[|stacklength\vb|]{|lap H-direction\vb|}{|item\vb|}|
\upshape

The names are simpler to remember, but they require, as their first
mandatory argument, the horizontal direction of lapping (l, c, or r),
while the material to be lapped is now in the second mandatory argument.
\vspace{-0.8ex}

\subsubsection{\cmd{stackanchor}\label{s:sanc}}

The \vb|\stackanchor| macro is one way to create a stack where the
baseline is split between two items in the stack.  The syntax is

\itshape
\vb|\stackanchor[|stackgap%
  \vb|]{|top item%
  \vb|}{|bottom item%
  \vb|}|
\upshape

The stacking gap may be provided as an optional argument.  All other
parameters taken are the defaults.  There are subtle differences in the
way a short stack anchor versus a long stack anchor is split.
One should consult \S\ref{s:sa} for a description of how the
stack is constructed.

If a short stack is being created with a split anchor, the
\vb|\strutshortanchors{}| mode setting will determine whether the split 
occurs at the mid-height of the strutbox (``T'') or whether it will 
occur about the baseline (``F'').

If a long stack is to be created with a split anchor, it is best to
formulate and \vb|\savestack| both the upper portion (with a stack)
and the lower portion (with an understack), and then \vb|\stackanchor|
the two parts together.  Alternately, one could use the 
\vb|\Centerstack| macro (\S\ref{s:csvs}).

\subsubsection{\cmd{Centerstack} and the math variants of 
  \cmd{Vectorstack}\label{s:csvs}}

The macro \cmd{Centerstack} produces a type of \cmd{Longstack} that is 
vertically centered about the mid-strut above the baseline.  In essence, 
it is like a \cmd{stackanchor} version of a \cmd{Longstack}.  Additionally, 
it is locally processed with the mode \cmd{strutlongstacks} set as 
\verb|T|, so as to ensure
the proper vertical centering.  

The \vb|\Centerstack| macro has a math-mode equivalent called 
\cmd{Vectorstack}, in which the stacking argument 
is automatically processed in math mode, regardless if whether the \cmd{stackMath} 
mode is currently active.

The syntax is the same as \vb|\Longstack|, except that the anchor is
not specified, but results naturally from the centering operation:

\itshape
\vb|\Centerstack[|alignment%
  \vb|]{|item %
  \vb|{|it em%
  \vb|} {\|item%
  \vb|} $|item%
  \vb|$| ... item%
  \vb|}|\\
\vb|\Vectorstack[|alignment%
  \vb|]{|item %
  \vb|{|it em%
  \vb|} {\|item%
  \vb|}| ... item%
  \vb|}|
\upshape

One of the main purposes of providing this vertically centered stack is
to be compatible with mathematical bracketing commands embodied in 
the use of the \LaTeX{} syntax \vb|\left(...\right)|, which require 
vertically centered material.  To this end, there
are several variants of \vb|\Vectorstack| macro that are for specific
use in mathmode.  Here are examples of all of these macros demonstrated:

{\centering
\Centerstack{p q r s}
\Vectorstack{p q r s}
$\parenVectorstack{p q r s}$
$\bracketVectorstack{p q r s}$
$\braceVectorstack{p q r}$
$\vertVectorstack{p q r}$.\par}
which are, respectively,\\
\vb|\Centerstack{p q r s}|\\
\vb|\Vectorstack{p q r s}|\\
\vb|$\parenVectorstack{p q r s}$|\\
\vb|$\bracketVectorstack{p q r s}$|\\
\vb|$\braceVectorstack{p q r}$.|\\
\vb|$\vertVectorstack{p q r}$.|


\subsubsection{\cmd{ensurestackMath}\label{s:esm}}

The macro \vb|\ensurestackMath{}| macro is used to force its argument
(a stacking command) to be processed in math mode, without setting 
the global default stacking 
mode to math mode (otherwise accomplished via \vb|\stackMath|). Note 
that this macro does \textbf{not} ensure that
all elements of its argument are processed in math mode (\textit{i.e.}, 
it does not use, nor is it a replacement for \vb|\ensuremath|), but 
only that stacking macros are processed in math mode.

\subsubsection{\cmd{abovebaseline} and \cmd{belowbaseline}\label{s:abb}}

The macros \vb|\abovebaseline| and \vb|\belowbaseline| also creates the
means to shift an item's baseline.  The syntax is

\itshape
\vb|\abovebaseline[|stackgap\vb|]{|item\vb|}|\\
\vb|\belowbaseline[|stackgap\vb|]{|item\vb|}|
\upshape

where the stackgap defines the stackgap distance.  In a long stack, it
will define the upward (for \vb|\abovebaseline|) or downward shift
(for \vb|\belowbaseline|) of the baseline of the item.  In a short
stack, it will define the upward/downward vertical gap between the
original baseline and the bottom/top of the shifted item.

For long stacks, a negative stackshift for \vb|\belowbaseline| will
produce the same result as the corresponding positive stackshift for
\vb|\abovebaseline|.  But this will not be the case for short stacks.
One should consult \S\ref{s:bbl} for more details.

This command can also be employed to redefine the baseline of graphical
objects, which can provide a useful technique to achieve horizontal
alignment with other objects.  Negative stack gaps are valid (see
\S\ref{s:bbl} for examples).  While these macros will carry slightly 
more processing overhead than a simple \vb|\raisebox|, it may be the 
case that specifying the shift in terms of the baseline, rather than 
with respect to the current configuration, may make for more 
streamlined logic.

\subsubsection{\cmd{stackinset}\label{s:iis}}

The macro \vb|\stackinset| is for insetting
smaller items (\textit{e.g.}, images or text) inside of a larger background
items (\textit{e.g.}, images or text).  The syntax is

\itshape
\vb|\stackinset{|H-align%
  \vb|}{|H-offset%
  \vb|}{|V-align%
  \vb|}{|V-offset%
  \vb|}{|inset item%
  \vb|}{|anchor item\vb|}|
\upshape

In addition to providing the inset and background items as the last two
arguments, the inset may be vertically and horizontally offset relative
to various references.  The first argument \textit{H-align}, is either
an \vb|l|, \vb|c|, or \vb|r|, indicating whether the horizontal offset
is relative to the left, center, or right of the anchor.  The second
argument is a length corresponding to the actual horizontal offset, 
relative to the horizontal alignment.  For \vb|l| and \vb|c| alignments, 
a positive horizontal offset is rightward, whereas it is leftward for
a \vb|r| alignment.

In the case of vertical measures, the third argument, \textit{V-align}, 
can have values of \vb|t|, \vb|c|, or \vb|b|, corresponding to top,
center, and bottom, respectively.  The fourth argument is a length
corresponding to the vertical offset from the alignment location.
For \vb|b| and \vb|c| alignments, a positive vertical offset is upward,
whereas it is downward for a \vb|t| alignment.

For horizontal and vertical offsets, negative lengths are allowed.
However, this macro is always executed with the \vb|\usestackanchor| 
parameter set as true (meaning that the resultant stack will be the
horizontal size of the anchor, even if the inset spills outside of it).

While this command was designed with inlaid images in mind (see
\S\ref{s:ii}), there is nothing that prevents its use to inlay
character glyphs upon each other, as in 
\stackinset{c}{}{c}{-.25ex}{*}{O}, accomplished with
\vb|\stackinset{c}{}{c}{-.25ex}{*}{O}|.

\subsubsection{\cmd{addstackgap}\label{s:asg}}
\def\stacktype{S}

The \vb|\addstackgap| macro provides a user-requested feature to add a
specified gap above and below an object.  The syntax is

\itshape
  \vb|\addstackgap[|%
  stackgap%
  \vb|]{|%
  item%
  \vb|}|
\upshape

When a stack is otherwise created, it by default will vertically extend
from the top of the top-most item in the stack to the bottom of the
bottom-most item in the stack, but with no extra gap applied.  Depending
on how the user employs that stack, there may not be enough vertical gap
between the stack and its adjacent objects.  This macro will apply a
fixed gaplength (either specified or, by default, an amount given by
\vb|\Sstackgap|) above and below the argument of the macro.  Negative
gaps will have no effect on the vertical extent of the item.

Here's an example of a stack, before and after \vb|\addstackgap| is
applied, framed for clarity:
\vb|\stackon{g}{a}|: \fbox{\stackon{g}{a}} versus
\vb|\addstackgap{\stackon{g}{a}}| \fbox{\addstackgap{\stackon{g}{a}}}.
This macro provides a slightly different result than the
\vb|\strutlongstacks| mode described in \S\ref{s:sm}.  For more details
of the comparison, see \S\ref{s:ag}.

\subsubsection{\cmd{hsmash}\label{s:hsm}}
\def\stacktype{L}

The macro \vb|\hsmash| is a horizontal version of the popular
\vb|\smash| macro:

\itshape
  \vb|\hsmash{|item\vb|}|
\upshape

In this case, the result is centered over its zero-width anchor
position.  If one wanted left or right aligned versions of this macro,
the existing \LaTeX{} macros \vb|\rlap| and \vb|\llap| would suffice.

The primary utility of this macro is in applying \vb|\hsmash| to an
object (often wider than the \vb|\textwidth|), in order to extend the
object into the margins in a controllable manner, such as in the case of
these items below:

{\centering\hsmash{\fbox{\stackon[3pt]{%
  \rule{1.1\textwidth}{.1ex}}{wide object}}}\par}

\hsmash{\rule{5ex}{1ex}}\hfill\hsmash{\rule{2ex}{0ex}\rule{5em}{1ex}}

\subsubsection{\cmd{savestack}\label{s:ss}}

The macro \vb|\savestack| provides a means to save an intermediate
result (in a box that can be recalled with the specified macro) without
printing it.  Its syntax is

\itshape
\vb|\savestack{|macro token\vb|}{|stacking operation\vb|}|
\upshape

The macro token is a backslashed name that you can later use to recall
the boxed intermediate result.  The stacking operation is the
intermediate result you wish to save in a box without printing.  Reasons
why this might be a useful macro are described in \S\ref{s:svs}.

Note that while \vb|\savestack| is designed to save an intermediate
stacking operation in a box, the reality is that you can use this macro
to save any sequence of boxable output, even if it is not a stacking
operation.  So, for example, \vb|\savestack{\mymacro}{\fbox{A}}| can
later be recalled with an invocation of \vb|\mymacro|, to yield
\fboxsep=1pt\savestack{\mymacro}{\fbox{A}}\mymacro.\fboxsep=0pt

\section{Making a Stack in Stages}

There are times where one needs to make a stack in several consecutive
steps.  In order to do so, there are several ways, detailed below.  

\subsection{Suppressing Output\label{s:qs}}

One can define \vb|\quietstack| as \vb|T| to suppress output.  Then, the
intermediate stack can be passed as an argument to the next stacking
operation with a \vb|\usebox{\stackedbox}|.  That same \vb|\usebox|
may be used to output the final stack, as long as \vb|\quietstack|
remains defined as ``T''.  Otherwise, \vb|\quietstack| must again be
defined ``F'' (false) to re-enable automatic output.

\subsection{Saving a Stack without Printing It\label{s:svs}}

Perhaps a quicker and better way is to use the package's
\vb|\savestack| macro, of the form
\itshape\vb|\savestack{|macro token\vb|}{|stacking operation\vb|}|
\upshape.  This will perform the stacking operation and place it into
the supplied token (without printing it), so that the intermediate stack
can be passed to the subsequent stacking operation by executing the
token macro.

For example,
\begin{verbatim}
\savestack{\pdq}{\Longstack[r]{6 5 4 {3 3}}}%
\Longstack[l]{{\pdq} 2 1 0}%
\end{verbatim}

would first place ``\Longstack[r]{6 5 4 {3 3}}'' into \vb|\pdq|, without
printing it, in a right-aligned fashion, with the ``3 3'' on the
baseline.  Then, the second \vb|\Longstack| would place the first stack
atop of the ``2'' in a left-aligned fashion.  The resulting stack would
look as follows, with the ``0'' on the final baseline.

\savestack{\pdq}{\Longstack[r]{6 5 4 {3 3}}}%
\Longstack[l]{{\pdq} 2 1 0}
\rl\rl\blmark%

\section{Choosing the Anchor Element\label{s:cae}}

Nearly all stacking done with the \textsf{stackengine} package has 
a designated anchor element.  The baseline of the anchor does not 
change.  Instead, the other items are placed relative to the anchor.
By adding items at first above and then below the anchor, any
arbitrary stack can be built around the anchor.

\def\stackalignment{r}
\setstackgap{S}{2pt}\setstackgap{L}{12pt}%
Here is a Longstack/understack example:
\savestack{\pdq}{\Longstack{3 2 1 0}}%
\Longunderstack[r]{{\pdq} -1 -2 -3 -4 -5}
that was achieved as follows:

\begin{verbatim}
\savestack{\pdq}{\Longstack{3 2 1 0}}%
\Longunderstack[r]{{\pdq} -1 -2 -3 -4 -5}
\end{verbatim}

In the first stack, ``0'' is the anchor of \vb|\pdq|.  In the second
stack, \vb|\pdq| is the anchor (which means that ``0'' remains the
anchor of the composite stack).

\section{Top and Bottom Lapping\label{s:lap}}
\def\stackalignment{c}
\def\stacktype{L}%
\setstackgap{L}{\baselineskip}
Top and bottom lapping can be accomplished with a long stack, using a null
item as the anchor, and \vb|\useachorwidth| set to \vb|T|.  This
technique is encoded in six \vb|\|\textit{xy}\vb|lap| commands,
where \textit{x} can be \vb|t| or \vb|b| (for ``top'' and
``bottom,'' respectively) and \textit{y} can be \vb|l|, \vb|c|, or
\vb|r| (for ``left,'' ``center,'' and ``right,'' respectively).  Note
that this \textit{y} is not the same as the \vb|\stackalignment|, but
rather refers to the direction where the lapped item is placed relative
to the lap invocation.

For illustration only, laps will be shown at a \textit{visible} lapmark
(given by \lapmark~), defining the point of invocation.  In these
examples, I lap a \vb|\parbox| above or below the selected location.

\begin{myverbbox}{\bll}\bllap\end{myverbbox}
\begin{myverbbox}{\brl}\brlap\end{myverbbox}
\begin{myverbbox}{\tll}\tllap\end{myverbbox}
\begin{myverbbox}{\trl}\trlap\end{myverbbox}
\begin{myverbbox}{\bcl}\bclap\end{myverbbox}
\begin{myverbbox}{\tcl}\tclap\end{myverbbox}
For example here\lapmark%
\bllap{\parbox[t]{.7in}{this is a bottom-left lap \bll}}
or el\lapmark%
\brlap{\parbox[t]{.8in}{this is a bottom-right lap \brl}}%
se here, or else I can put it here\lapmark%
\tllap{\parbox[b]{.7in}{this is a top-left lap \tll}}
or here\lapmark%
\trlap{\parbox[b]{.7in}{this is a top-right lap \trl}}%
.\

Of course, I can per\lapmark
\bclap{\fbox{\parbox[t]{1.1in}{this is a bottom-center lap \bcl}}}%
form center lapping, as we\lapmark%
\tclap{\fbox{\parbox[b]{1.1in}{this is a top-center lap \tcl}}}%
ll.
To see, for example, how the above was achieved, the last lap in the
prior sentence was achieved by invoking

\vb|\tclap{\fbox{\parbox[b]{1.1in}{this is a top-center lap \tcl}}}|

in the middle of the word ``well.''  (Note that \vb|\tcl| is a box
containing the verbatim text ``\tcl,'' created with the
\textsf{verbatimbox} package.)

Note that \vb|\toplap| and \vb|\bottomlap| have also been introduced
as macros that are functionally identical to these six lapping commands.
While requiring an extra argument, their names may be easier for you to
remember.  See \S\ref{s:lapdef}.\vspace{-1ex}

\section{Shifting the Anchor's Baseline} 

By default, the baseline of the anchor becomes the baseline of the
stack.  If one wanted the stack baseline to \textbf{not} conform to
the baseline of any of the individual stacked items, something needs
to change.  If one quantitatively knows the shift desired, a simple
\vb|\raisebox| could be used to vertically shift the anchor, after
(or possibly prior to) stacking.  There are also some other options
provided by the package.

\subsection{The \cmd{stackanchor} Macro for ``L'' and ``S'' Stack Types%
\label{s:sa}}

If one would like the anchor's baseline to be ``midway'' between two
vertically arrayed items, the \vb|\stackanchor| macro attempts to
provide that capability for both ``L'' and ``S'' stack types, though the
definitions of ``midway'' are different for each.  The macro
\vb|\stackanchor| works for two-item stacks in both ``L'' and ``S''
stack types, as shown below.

In the case on a long stack, what is satisfied is that the baseline of
the anchor is made equidistant from the baselines of the two respective
items comprising the stacked anchor.

\def\stacktype{L}%

\Large
\def\stackalignment{r}%
\setstackgap{L}{.8\baselineskip}%
\blmark%
\rl\rl
\stackanchor{\rl\rl\rl\rl top\rl}{\rl L-bottom\rl}%
\rl\rl%
\blmark
\normalsize

In the case of a short stack, the middle of the inter-item gap is
vertically located at the center of the font's \vb|\strutbox|, so that
the middle of the stackgap would be at the same vertical level as the
middle of parentheses, were they located on the original baseline.

\Large
\blmark\rl\rl%
\def\rls{\rule[.5\ht\strutbox-.5\dp\strutbox]{2ex}{.1pt}}%
\rls%
\def\stackalignment{l}%
\def\stacktype{S}\setstackgap{S}{1ex}%
\savestack{\pdq}{\stackanchor{\fbox{top}}{\fbox{S-bottom}}}%
\pdq%
\rls%
(Mid-strut)%
\rl\rl\blmark
\def\stackalignment{c}%
\normalsize%

If one does not want to have the short-stack gap split at the mid-strut
height, but rather at the baseline, the \vb|\strutshortanchors{F}| mode
may be set prior to the \vb|\stackanchor| invocation.

\strutshortanchors{F}%
\Large
\blmark\rl\rl%
\def\stackalignment{l}%
\def\stacktype{S}%
\savestack{\pdq}{\stackanchor{\fbox{top}}{\fbox{S-bottom}}}%
\pdq%
\rl\rl\blmark
\def\stackalignment{c}%
\normalsize%
\strutshortanchors{T}%

The \vb|\strutshortanchors| mode will only affect short stacks
(\vb|\stacktype| ``S''), since long stacks do not use a strut as part
of the \vb|\stackanchor| definition.

For stackanchors with more than two rows, one should compose the top
portion with a stack and the bottom portion with an understack, and
then, finally, shift the anchor by applying \vb|\stackanchor| to these
two ``substacks.''

\Large
\setstackgap{S}{1ex}%
\savestack{\upperstack}{\Shortstack{{over text} {\fbox{top}}}}
\savestack{\lowerstack}{\Shortstack{{\fbox{S-bottom}} {under text}}}
\blmark\rl\rl\rls%
\stackanchor{\upperstack}{\lowerstack}%
\rls(Mid-strut)
\normalsize

\subsection{\cmd{belowbaseline} for Moving the Anchor's
Baseline\label{s:bbl}}

The command \vb|\belowbaseline| is intended to be used to set an item,
such as a stack anchor, below the baseline (this discussion is equally
valid for the macro \vb|\abovebaseline|, but in the opposite sense).
The macro accomplishes this task by stacking the item below a null
object.  However, \vb|\useanchorwidth| is temporarily set to \vb|F| so
that the stacked object takes on the width of the under-placed object.

The command can be used with both stacktypes.  With \vb|\stacktype|
``S'', the optional argument will denote the stackgap below the baseline
(default \vb|\Sstackgap|),

\Large
\setstackgap{S}{6pt}%
\def\stacktype{S}%
Short stacktype\rl\rl\belowbaseline[0pt]{\fbox{0pt below baseline}}%
\rl\belowbaseline{\fbox{6pt}}\rl\belowbaseline[-6pt]{\fbox{-6pt}}\rl\rl
\normalsize

When used with \vb|\stacktype| ``L'', the optional length denotes
the distance to the baseline of the understacked object (default
\vb|\Lstackgap|).

\Large
\setstackgap{L}{18pt}%
\def\stacktype{L}%
Long stacktype\rl\rl\belowbaseline[0pt]{\fbox{0pt below baseline}}%
\rl\belowbaseline{\fbox{18pt}}\rl\belowbaseline[-19pt]{\fbox{-18pt}}\rl
\normalsize

As the above images show, there is nothing to prevent the use of
negative numbers being used to raise the the object above the baseline.
But care must be taken, since a \vb|\belowbaseline| with a negative
shift will not necessarily equal the \vb|\abovebaseline| with the
corresponding positive shift (that equality only holds for long stacks).
To better appreciate the meaning of negative baseline shifts, we
compare, below, negative shifts for \vb|\belowbaseline| to
\vb|\abovebaseline| with the corresponding positive argument.

\Large
\def\stacktype{S}%
Short stacktype\rl\rl\belowbaseline[-6pt]{\fbox{-6pt below}}%
\rl\abovebaseline[6pt]{\fbox{6pt above}}\rl\rl\blmark%
\normalsize

\Large
\def\stacktype{L}%
Long stacktype\rl\rl\belowbaseline[-6pt]{\fbox{-6pt below}}%
\rl\abovebaseline[6pt]{\fbox{6pt above}}\rl\rl\blmark%
\normalsize

The \vb|\belowbaseline| command is also useful for forcing objects,
such as images, to be top-aligned.  This can be a useful feature when
trying to achieve a desired horizontal alignment of disparate elements in
tabular columns.

\vb|\def\stacktype{S}\belowbaseline[0pt]{\imgi}|\\
\vb|\belowbaseline[-\ht\strutbox]{\imgi}|:

\def\stacktype{S}
\def\imgi{\includegraphics[width=0.87in]{example-image}}
\rl\blmark\rl\rl\belowbaseline[0pt]{\imgi}\rl\rl
\belowbaseline[-\ht\strutbox]{\imgi}\rl\rl%
\fbox{\parbox[t]{1.8cm}{\raggedright A parbox with [t] alignment}}%
\rl\rl\blmark\rl

\section{Negative Stack Gap\label{s:nsg}}
\def\stacktype{S}
\setstackgap{S}{-1.2pt}
\setlength{\dotscale}{1.ex}
A negative gap can be used to overlap things.  For example, \newdot plus
``i'' can make, when the stack gap is suitably negative, the following:

~\hfill\scalebox{3}{\stackon{i}{\newdot}}\hfill~

Note that the use of a negative stack gap for a two-item stack is
actually very similar to the \vb|\stackinset| macro described in
\S\ref{s:iis} and \S\ref{s:ii}.  The issue of negative stackgaps is also
discussed in \S\ref{s:bbl}.

\section{The {\cmd{useanchorwidth}} Option\label{s:uaw}}
\setstackgap{S}{-1.2pt}
\setlength{\dotscale}{1.ex}

Because\newdot has width larger than the dot glyph itself (we box its
extent here: \fbox{\newdot}), stacking it could otherwise be a problem.
For example, th\stackon{i}{\newdot}s uses the whole stackwidth to set
the dotted-i in its place. 
\def\useanchorwidth{T} 
On the other hand, when \vb|\useanchorwidth| is defined as \vb|T|,
``th\stackon{i}{\newdot}s'' uses only the anchor width to set the stack
(where the ``i'' is the anchor).  

Below, we show an \vb|\fbox| around the dotted i, clearly
demonstrating that the dot does not influence the width of the
stack when \vb|\useanchorwidth| is defined as ``T''.

\fboxrule=0.2pt
\hfill 
\scalebox{3}{th\fbox{\stackon{i}{\newdot}}s}%
\hfill~
\normalsize
\def\useanchorwidth{F} 
\fboxrule=0.6pt

\section{Adding Gap Above and Below a Stack\label{s:ag}}
\setstackgap{L}{\baselineskip}

By default, a stack is vertically clipped with no gap as in these
two \vb|\Longstack| examples:
\fbox{\Longstack{A G}}
\fbox{\Longstack{a g}}

This package provides two ways to conveniently add gaps above and below
a stack.  For any type of stack (short or long), the \vb|\addstackgap|
macro (\S\ref{s:asg}) adds a fixed amount of gap above and below a
completed object.  Alternately, during the construction of long stacks,
the \vb|\strutlongstacks| mode (\S\ref{s:sm}) will automatically employ
struts in each row of a long stack during its construction.  However,
these two approaches are not equivalent:

\vb|\addstackgap[3pt]{}|:
\fbox{\addstackgap[3pt]{\Longstack{A G}}}
\fbox{\addstackgap[3pt]{\Longstack{a g}}}
~~~~\vb|\strutlongstacks{T}|:
\strutlongstacks{T}
\fbox{\Longstack{A G}}
\fbox{\Longstack{a g}}
\strutlongstacks{F}

Make sure you use the one that best applies to your situation.

\section{Image Insets\label{s:ii}}
\setstackgap{L}{18pt}%

The command \vb|\stackinset| allows images to be
stacked atop one another, to produce the effect of an inset image.  In
addition to specifying the two images, the effect can be tailored by
specifying the vertical and horizontal offset of the inset image.

\def\imgi{\includegraphics[width=0.87in]{example-image}}
\def\imgii{\includegraphics[width=0.3in]{example-image}}
\savestack{\imga}{\stackinset{r}{3pt}{t}{2pt}{\imgii}{\imgi}}
\savestack{\imgb}{\stackinset{l}{3pt}{b}{2pt}{\imgii}{\imgi}}
\imga~~~~\imgb
\def\stackalignment{c}

The last figure was obtained with
\vb|\stackinset{l}{3pt}{b}{2pt}{\imgii}{\imgi}|,
showing that \vb|\imgii| was offset 2pt from the bottom and 3pt from
the left edge of \vb|\imgi|.

The syntax of \vb|\stackinset| allows for a convenient nesting of
figure text labels.  For example,

\begin{verbatim}
\stackinset{l}{ .1in}{b}{.2in}{Note 1}{%
\stackinset{r}{ .1in}{t}{.2in}{Note 2}{%
\stackinset{c}{-.5in}{c}{.4in}{Note 3}{%
\stackinset{r}{ .8in}{b}{.6in}{Note 4}{%
\stackinset{r}{ .1in}{c}{.0in}{\imgi}{%
\includegraphics[width=3in]{example-image}%
}}}}}
\end{verbatim}

gives the following result:\begin{verbbox}[\small]\stackinset\end{verbbox}

\bxfigure[ht]
{An example of \theverbbox}
{%
\stackinset{l}{ .1in}{b}{.2in}{Note 1}{%
\stackinset{r}{ .1in}{t}{.2in}{Note 2}{%
\stackinset{c}{-.5in}{c}{.4in}{Note 3}{%
\stackinset{r}{ .8in}{b}{.6in}{Note 4}{%
\stackinset{r}{ .1in}{c}{.0in}{\imgi}{%
\includegraphics[width=3.0in]{example-image}%
}}}}}%
}

\clearpage
\section{Stacking for Subfiguring and Baseline Manipulation\label{s:bm}}

The stacking commands provide a convenient way to underset a subfigure
notation under the corresponding image.  Furthermore, because of the
flexibility of the stacking commands, the baselines of the subfigures
can be manipulated in many ways, which may come in handy depending on
how you are choosing to lay out your subfigures.

Examples of this are shown in figure~\ref{fg:subs}. 

%
\setstackgap{S}{3pt}\def\stacktype{S}%
\begin{myverbbox}{\figa}\stackon{(a)}{\imga}\end{myverbbox}%
\begin{myverbbox}{\figb}\stackunder{\imgb}{(b)}\end{myverbbox}%
\begin{myverbbox}[\small]{\figc}
\Shortstack{{\imgi} (c1) {} {\imgi} (c2)}
\end{myverbbox}%
\begin{myverbbox}%
{\figd}\belowbaseline[0pt]{\stackunder{\imgi}{(d)}}\end{myverbbox}%
\def\CaptionJustification{\raggedright}%
\bxfigure[ht]
{Image stacking with stacktype ``S'', stack gap set to 3pt and using: 
(a)~\figa, (b)~\figb, (c)~\figc, and (d)~\figd\label{fg:subs}}
{\blmark\rl\rl%
\stackon{(a)}{\imga}~~
\stackunder{\imgb}{(b)}~~
\Shortstack{{\imgi} (c1) {} {\imgi} (c2)}~~
\belowbaseline[0pt]{\stackunder{\imgi}{(d)}}
}%

\section{Nested Stacking Commands\label{s:nsc}}

The macros \vb|\Shortstack|, \vb|\Longstack|, and the corresponding
understack macros are set up to conveniently stack multiple objects
using a space separated argument list.  Because of parsing constraints
on the space-separated list, however (see \S\ref{s:mm}), there may be
times that you would prefer building stacks by way of nesting the more
primitive stacking macros.

Nesting can be accomplished, but some care must be taken, in the case of
long stacks.  But first, let us see the ways in which nesting may be
applied for short stacks:\\

\def\stacktype{S}\setstackgap{S}{3pt}
\vb|\stackunder{1}{\stackunder{2}{\stackunder{3}{\stackunder{4}{5}}}}|\\
\vb|\stackunder{\stackunder{\stackunder{\stackunder{1}{2}}{3}}{4}}{5}|\\
\vb|\stackon{1}{\stackon{2}{\stackon{3}{\stackon{4}{5}}}}|\\
\vb|\stackon{\stackon{\stackon{\stackon{1}{2}}{3}}{4}}{5}|

will produce the following: 
\stackunder{1}{\stackunder{2}{\stackunder{3}{\stackunder{4}{5}}}}
\stackunder{\stackunder{\stackunder{\stackunder{1}{2}}{3}}{4}}{5}
\stackon{1}{\stackon{2}{\stackon{3}{\stackon{4}{5}}}}
\stackon{\stackon{\stackon{\stackon{1}{2}}{3}}{4}}{5} \blmark

In these cases, the nesting could be applied on either the first or the
second argument of the \vb|\stackunder| or \vb|\stackon| commands,
with the same end result.

In the case of long stacks however, employing the identical methodology
gives a different, at first unexpected, result:
\def\stacktype{L}\setstackgap{L}{\baselineskip}
\stackunder{1}{\stackunder{2}{\stackunder{3}{\stackunder{4}{5}}}}
\stackunder{\stackunder{\stackunder{\stackunder{1}{2}}{3}}{4}}{5}
\stackon{1}{\stackon{2}{\stackon{3}{\stackon{4}{5}}}}
\stackon{\stackon{\stackon{\stackon{1}{2}}{3}}{4}}{5} \blmark

In this case, nesting the second argument works as hoped, but nesting
the first argument does not.  The reason for this ``failure'' stems
not from a failure of logic, but an enforcement of it.  It stems from the
very definition of a long stack, in which the baseline of the second
argument is placed \vb|\LStackgap| below or above the baseline of the
first argument.  By nesting the first argument, all the second arguments
are placed \vb|\Lstackgap| from the unchanging stack baseline (thus
overlapping), and not relative to the current top or bottom of the
nested stack.

Therefore, to nest long stacks, only the second argument of the
stackengine may be nested.

\section{Math Mode and Mixed Mode Usage\label{s:mm}}
\def\useanchorwidth{F}
\setstackgap{S}{1pt}\def\stacktype{S} 

As of V3.0, the {\ste} package was extended to allow its arguments to be
processed, by default, using inline math-mode (\vb|\textstyle| math).
The math-mode default is brought about with an invocation of
\vb|\stackMath|, which can be later reversed via \vb|\stackText|.

Except in circumstances that require the particular features of this
package, {\ste} may not be the best package for math mode
since there are many packages that already cater directly to the need to
stack and align mathematical objects.  However, if there is a need to
use {\ste} there, because of its particular facility with
stacking gaps and alignment, there are several hints to remember.

\textsc{Hints for use in math mode:} (operating under \vb|\stackMath|)
\stackMath

\begin{enumerate}

\item Arguments to stacking commands are only taken in math mode, by
default, following the invocation of the \vb|\stackMath| macro.

\item The \vb|\stackMath| and \vb|\stackText| macros are not to be
used within stacking arguments, but are invoked prior to stacking, to
set the default manner in which stacking arguments are subsequently
processed.

\item As described in \S\ref{s:sls}, if the math argument has spaces in
a command which uses space-separated lists, or if the argument ends in a
macro, the argument must be enclosed in braces.

\vb|\(y = \Shortstack{{a + b} {\odot} b} x\)|:\hfill
\(y = \Shortstack{{a + b} {\odot} b} x\)

\item In math mode, the \vb|\stackanchor| command seems to preserve
the math axis better with an ``S'' \vb|\stacktype|, rather than type
``L''.

\end{enumerate}

\textsc{Hints for using math in text mode:} (operating under \vb|\stackText|)
\stackText

%SAVE FOR FUTURE USE
\begin{verbbox}[]
\savestack{\a}{\(\sqrt{a + b}\)}
\savestack{\dash}{- - - - -}
\(y = \stackunder{\stackon{\dash}{\a}}{$b$} x\)
\end{verbbox}

\begin{enumerate}

\item Stacking commands from this package can be used in math mode, but
will set their arguments as text, 

\vb|\( y = \stackunder{a}{+} x\)|:\hfill
\( y = \stackunder{a}{+} x\)

\item To set an argument in math style, it must be set between math
delimiters.

\vb|\( y = \stackunder{$a$}{+} x\)|:\hfill
\( y = \stackunder{$a$}{+} x\)

\item To set a particular stack completely in math style, without globally setting the mode with a \verb|\stackMath|, one can use \verb|\ensurestackMath{}|.

\item If not using \verb|\ensurestackMath|, each space-separated item of the argument to \vb|\Shortstack|,
\vb|\Longstack|, and the corresponding understacks must each be enclosed
in their own set of math delimiters (either \verb|$...$| or \verb|\(...\)|).

\vb|\(y = \Shortstack{$a$ . $b$} x\)|:\hfill
\(y = \Shortstack{$a$ . $b$} x\)

\item For more complex items, the \vb|\savestack| command can be used
to save an intermediate math expression (even if it isn't actually a
stack).

\theverbbox:\hfill
\savestack{\a}{\(\sqrt{a + b}\)}
\savestack{\dash}{- - - - -}
\addvbuffer[\the\baselineskip]{\(y = \stackunder{\stackon{\dash}{\a}}{$b$} x\)}

\end{enumerate}

\section{Deprecated Features\label{s:df}}

There are several definitions and macros from earlier versions of the 
{\ste} package that are still fully functional, but are discouraged 
because of more preferable alternatives.  They have been omitted from the
current documentation, except for mention here.  They are:

\begin{itemize}
\item \vb|\def\usestackstrut{T/F}|.  This definition is fully replaced 
by the mode setting given by the \vb|\strutshortanchors{T/F}| macro.
The new name is less ambiguous and the macro form is chosen vis-\`a-vis
the \vb|\def| form, because the macro represents a \textbf{mode} of
stack construction, rather than an argument parameter.

\item \vb|\bottominset| and \vb|\topinset|.  These
macros are subsumed by the much more versatile \vb|\stackinset| macro.
The replacement offers insets relative to not only the top and bottom,
but also the mid-height of the anchor.  It also provides the capability 
to horizontally place the inset relative to the center of the anchor,
in addition to the left or the right.  Most importantly, the placement
of the anchor as the last argument of \vb|\stackinset| means that the
syntax for nested insets is much more readable vis-\`a-vis the nested
syntax for \vb|\bottominset| and \vb|\topinset|.
\end{itemize}

\section{Backward Compatibility\label{s:bc}}

The author of this package profoundly apologizes for the fact that the
syntax of this package changed between versions 1.0 and 2.0 of this
package.  

In\margcmd{setstackgap} version 1.0 of this package, stackgaps could be
directly defined by setting the \LaTeX{} lengths \vb|\Sstackgap| and
\vb|\Lstackgap|.  The problem with this approach is that, under fontsize
changes, these stackgaps were not automatically updated to reflect the
new fontsize.  As of V2.0, that deficiency was remedied.  The syntax
that replaces these length specifications is

\itshape
\vb|  \setstackgap{S}{|inter-item stackgap\vb|}|\\
\vb|  \setstackgap{L}{|inter-item baselineskip\vb|}|  .
\upshape

Thus, \vb|\Sstackgap| and \vb|\Lstackgap| are no longer lengths (but
instead \vb|\def|s), and cannot be set via the \vb|\Sstackgap=...|
or \vb|\setlength{\Sstackgap}{...}| syntax.  It is these settings
which will need to be revised to modernize older code.

While\margtt{[oldsyntax]} not encouraged, the V1.0 syntax may be
recovered via

\vb|  \usepackage[oldsyntax]{stackengine}|

If you retain the ``oldsyntax'' setting, you must remember that
\textsf{stackengine} will honor your specified gap sizes
(\vb|\Sstackgap| and \vb|\Lstackgap|) at the time of their
\textbf{definition}, rather than at the time of their
\textbf{invocation}.  Thus, if \vb|\Sstackgap| is specified as 0.7ex
when the text is \vb|\normalsize|, the gap length is set in a
\vb|\normalsize| font, even if the fontsize is later changed.  To
overcome this behavior, the stackgap must be redefined at the new
fontsize, or better still, \vb|[oldsyntax]| usage needs to be
relegated.

\clearpage
\section{Acknowledgements}

I would like to thank Prof. Enrico Gregorio at tex.stackexchange for
assisting in numerous ways to make this package better:

\begin{itemize}

\item by providing  three lines of code to strip a leading 
backslash from an argument:\\
\vb|http://tex.stackexchange.com/questions/42318/|\\
\vb|removing-a-backslash-from-a-character-sequence|

\item in making the
transition from the deficient V1.0 syntax as painless as possible:\\
\vb|http://tex.stackexchange.com/questions/123443/|\\
\vb|defining-a-length-that-scales-with-fontsize-changes/123470#123470|

\item and in helping to make the parsing routines of this package robust:\\
\vb|http://tex.stackexchange.com/questions/137298/|\\
\vb|centering-breaks-fragile-command-is-there-a-fix/137314#137314|

\end{itemize}

Thanks also to Dr. David Carlisle for explaining the need for argument
expansion in certain situations:\\
\vb|http://tex.stackexchange.com/questions/131581/|\\
\vb|parsing-rows-and-tab-characters/131594#131594|

Finally, I extend my thanks to Per Starb\"ack, for 
his unsolicited but gratefully accepted edit of the V3.21
package documentation, which corrected numerous typographical errors.

\section{Sectional index of \textsf{stackengine} definitions, modes, 
  \& macros\label{s:index}}

{\small
\begin{tabular}{lll}
Definition, Mode, or Macro & \S\ Defined & \S\ Used\\
\hline
\cmd{abovebaseline}&\ref{s:abb}&\ref{s:bbl}\\
\cmd{addstackgap}&\ref{s:asg}&\ref{s:sou}, \ref{s:ag}\\
\cmd{bclap}&\ref{s:lapdef}&\ref{s:lap}\\
\cmd{belowbaseline}&\ref{s:abb}&\ref{s:bbl}\\
\cmd{bllap}&\ref{s:lapdef}&\ref{s:lap}\\
\cmd{bottomlap}&\ref{s:lapdef}&\ref{s:lap}\\
\cmd{braceVectorstack}&\ref{s:csvs}&\\
\cmd{bracketVectorstack}&\ref{s:csvs}&\\
\cmd{brlap}&\ref{s:lapdef}&\ref{s:lap}\\
\cmd{Centerstack}&\ref{s:csvs}&\ref{s:sanc}\\
\cmd{ensurestackMath}&\ref{s:esm}&\ref{s:sm}, \ref{s:mm}\\
\cmd{hsmash}&\ref{s:hsm}&\\
\cmd{Longstack}&\ref{s:sls}&\ref{s:st}, \ref{s:osus}, \ref{s:sm}, \ref{s:slus}, 
  \ref{s:csvs}, \ref{s:svs}, \ref{s:cae}, \ref{s:ag}, \ref{s:nsc},
  \ref{s:mm}\\
\cmd{Longunderstack}&\ref{s:slus}&\ref{s:osus}, \ref{s:sm}, \ref{s:cae},
  \ref{s:nsc}, \ref{s:mm}\\
\cmd{Lstackgap}&\ref{s:parms}&\ref{s:st}, \ref{s:stackeng}, \ref{s:bbl},
  \ref{s:nsc}, \ref{s:bc}\\
\cmd{parenVectorstack}&\ref{s:csvs}&\\
\cmd{quietstack}&\ref{s:parms}&\ref{s:stackeng}, \ref{s:qs}\\
\cmd{savestack}&\ref{s:ss}&\ref{s:sanc}, \ref{s:svs}, \ref{s:cae},
  \ref{s:mm}\\
\cmd{setstackEOL}&\ref{s:sm}&\ref{s:sls}\\
\cmd{Shortstack}&\ref{s:sls}&\ref{s:st}, \ref{s:osus}, \ref{s:anch},
  \ref{s:sal}, \ref{s:sm}, \ref{s:slus},  \ref{s:sa}, \ref{s:bm}, \ref{s:nsc},
  \ref{s:mm}\\
\cmd{Shortunderstack}&\ref{s:slus}&\ref{s:st}, \ref{s:osus}, \ref{s:anch},
  \ref{s:sm}, \ref{s:sls}, \ref{s:nsc}, \ref{s:mm}\\
\cmd{Sstackgap}&\ref{s:parms}&\ref{s:st}, \ref{s:stackeng}, \ref{s:asg},
  \ref{s:bbl}, \ref{s:bc}\\
\cmd{stackalignment}&\ref{s:parms}&\ref{s:sal}, \ref{s:stackeng},
  \ref{s:cae}, \ref{s:lap}\\
\cmd{stackanchor}&\ref{s:sanc}&\ref{s:csvs}, \ref{s:iis}, \ref{s:sa},
  \ref{s:mm}\\
\cmd{stackengine}&\ref{s:stackeng}&\ref{s:sou}\\
\cmd{stackgap}&\ref{s:parms}&\ref{s:stackeng}, \ref{s:sls}\\
\cmd{stackinset}&\ref{s:iis}&\ref{s:nsg}, \ref{s:ii}, \ref{s:df}\\
\cmd{stackMath}&\ref{s:sm}&\ref{s:sls}, \ref{s:esm}, \ref{s:mm}\\
\cmd{stackon}&\ref{s:sou}&\ref{s:asg}, \ref{s:nsg}, \ref{s:uaw}, \ref{s:bm},
  \ref{s:nsc}, \ref{s:mm}\\
\cmd{stackText}&\ref{s:sm}&\ref{s:mm}\\
\cmd{stacktype}&\ref{s:parms}&\ref{s:st}, \ref{s:stackeng}, \ref{s:sls},
  \ref{s:sa}, \ref{s:bbl}, \ref{s:bm}, \ref{s:mm}\\
\cmd{stackunder}&\ref{s:sou}&\ref{s:bm}, \ref{s:nsc}, \ref{s:mm}\\
\cmd{strutlongstacks}&\ref{s:sm}&\ref{s:csvs}, \ref{s:asg}, \ref{s:ag}\\
\cmd{strutshortanchors}&\ref{s:sm}&\ref{s:sanc}, \ref{s:sa}, \ref{s:df}\\
\cmd{tclap}&\ref{s:lapdef}&\ref{s:lap}\\
\cmd{tllap}&\ref{s:lapdef}&\ref{s:lap}\\
\cmd{toplap}&\ref{s:lapdef}&\ref{s:lap}\\
\cmd{trlap}&\ref{s:lapdef}&\ref{s:lap}\\
\cmd{useanchorwidth}&\ref{s:parms}&\ref{s:stackeng}, \ref{s:bbl}, \ref{s:uaw}\\
\cmd{Vectorstack}&\ref{s:csvs}&\\
\cmd{vertVectorstack}&\ref{s:csvs}&\\
\end{tabular}
}

\section{Code Listing}

\verbfilenobox[\footnotesize]{stackengine.sty}

\end{document}



